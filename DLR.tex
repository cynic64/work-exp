\documentclass[12pt]{article}

\usepackage[margin=1.0in]{geometry}
\usepackage{xcolor}
\usepackage{ulem}
\usepackage{microtype}
\usepackage{babel}

% title
\title{\vspace{-96pt} \Huge Lebenslauf}
\author{\Huge Nicholas Schweitzer \vspace{-32pt}}
\date{}

% link and section macros
\newcommand{\link}[1]{{\color{blue}\underline{#1}}}
\newcommand{\sect}[1]{\vspace{-12pt}\section*{#1}\vspace{-12pt}}

% page formatting options
\setlength\parindent{0pt}
\setlength\parskip{6pt}
\pagenumbering{gobble}

% other commands
\newcommand{\sep}{{\color{gray}\vspace{-12pt}\hrule}}
\def\labelitemi{--}
\newcommand{\ask}[1]{{\color{red}#1}}

% main document
\begin{document}
\maketitle

% \section*{Motivation}
% Ich bin Sch{\"u}ler in die 10. Klasse und bin in ein Praktikum bei DLR entweder in Oberpfaffenhofen oder Weilheim im Bereich von Aerodynamik und Raumfahrt interessiert. Technologie, Physik und Mathematik fand ich immer interessant und ich hoffe durch ein Praktikum bei DLR mein Wissen zu erweitern und praktische Erfahrung zu gewinnen sowie mit anderen {\"u}ber meine Interessen reden.

\sect{Pers{\"o}nliche Daten}

\textbf{Geburt} \hfill{27.09.2003} \, \\
\sep
\textbf{Email} \hfill{nicholas\_schweitzer@mis-munich.de} \, \\
\sep
\textbf{Handy / Festnetz} \hfill{0176 8053 1248 / 08171 238 1130} \, \\
\sep
\textbf{Adresse} \hfill{Am Haken 2, 82538 Geretsried \, \\
  \sep

  \sect{Profil}
  \begin{itemize}
    \itemsep3pt

  \item Schuler Schwerpunkt in Mathematik und Physik mit Kursen in \textit{\glqq
    Advanced Mathematics\grqq} und \textit{\glqq Extended Science\grqq}
    (fortgeschrittene Mathematik und erweiterte Physik, Chemie und Biologie)

  \item Programmieren seit 7 Jahren als Hobby mit fortgeschrittenen Kentnissen in
    Python und Rust

  \item Musik und Sport: Gitarre seit 10 Jahren als intensives Hobby,
    Tischtennis Vereinsspieler

  \item Sonstige Interessen: Freiwillige Feuerwehr und Sportklettern

  \end{itemize}
  \vspace{-12pt}

  \sect{Schullaufbahn}

  \textbf{Seit August 2018: \\
    Senior School der Munich International School, aktuell in der 10. Klasse
  }

  \vspace{-10pt}
  \begin{itemize}
    \itemsep3pt
  \item Angestrebter Abschluss: International Baccalaureate im Juni 2022

  \item \glqq Head of School's List\grqq \,in die 5., 6., 7., 8. und 9. Klasse
    (Auszeichnung f{\"u}r die Jahrgangsbesten mit einer Durchscnittsnote von
    mindestens 6,5 auf eine Skala von 0,0 bis 7,0)

  \item Fachliche Schwerpunkte und zus{\"a}tzliche Kurse:

    \begin{itemize}
      \itemsep0pt
      \vspace{-12pt}
    \item \glqq Advanced Math\grqq: Fortgeschrittene Mathematik

    \item \glqq Science Extension\grqq: erweiterte Physik, Chemie und Biologie, 3 Stunden pro Woche

    \item Musik, 3 Stunden pro Woche
    \end{itemize}

  \item 3D-Renderer programmiert f{\"u}r den \glqq Personal project\grqq

  \item Schuljahrbegleitendes \textit{\glqq Personal project\grqq}:
    Programmieren einen 3D-Renderer in Rust. \ask{More description meaning more
      about the PP itself or my renderer?}

  \item Live-Demonstration f{\"u}r eine Veranstaltung mit Eltern und Sponsoren
    der Schule: bewegungsbesteuerter Theremin f{\"u}r die Schule erstellt
    (Eltern k{\"o}nnten durch Bewegung vor ein Green-Screen den Ton und
    Lautst{\"a}rke eines virtuellen Instrumentes kontrollieren)
  \end{itemize}

  \textbf{2008 - 2017:
  }
  \vspace{-10pt}
  \begin{itemize}
    \itemsep3pt
  \item Kindergarten und 1. Klasse bei Munich International School

  \item Homeschooling f{\"u}r die 2., 3. und 4. Klasse wegen einen dreijahrigen Segelurlaub mit Eltern

  \item 5. bis 8. Klasse: Middle School bei Munich International School
  \end{itemize}
  % 6,50; 6,56; 6,56; 6,67; 6,56;

  % \section*{Motivation}
  %
  % Ich bin ein Sch{\"u}ler in die 10. Klasse der Munich International School, eine internationale IB-Schule. Meine Interesse in Technologie, Physik und Mathematik, besonders im Bezug auf Aerodynamik und Raumfahrt, hat mit zu DLR gef{\"u}hrt. Ich hoffe durch ein Praktikum in diesen Bereich (Aerodynamik und Raumfahrt) entweder im Oberpfaffenhofen oder Weilheim DLR-Zentrum mein Wissen zu erweitern und praktische Erfahrung zu gewinnen.

  \sect{Kenntnisse}
  \textbf{Sprachen} \hfill{Deutsch} \\
  \hspace*{\fill}Englisch, flie{\ss}end \\
  \hspace*{\fill}Spanisch, 4 Jahren (Phase 3/4) \\
  \sep

  \vspace{\parskip}

  \textbf{Programmiersprachen} \\
  \textit{Sehr gut} \hfill{Python, Rust} \\
  \sep
  \textit{Gut} \hfill{C/C++, \LaTeX, Shell (bash/zsh), Lua} \\
  \sep
  \textit{Grundkentnisse} \hfill{Lisp, HTML, Perl, Intel-Assembly (NASM)} \\
  \sep

  \vspace{\parskip}

  \textbf{Github-Repos} \hfill{3D-Renderer in Rust (\link{github.com/cynic64/render-engine})} \\
  \hspace*{\fill} Bewegungsbesteuerter Theremin (\link{github.com/cynic64/theremin}) \\
  \hspace*{\fill} Window-Manager Konfiguration (\link{github.com/cynic64/awesome-configs}) \\

  \vspace{4pt}
  \sep

  \vspace{\parskip}

  \textbf{Software / Platforme} \\
  \textit{Sehr gut} \hfill{Emacs, Blender, Word} \\
  \sep
  \textit{Gut} \hfill{Github, Linux-Administration, Word, Excel, PowerPoint} \\
  \sep

  \vspace{\parskip}
  \sect{Mitgliedschaften}

  \textbf{2018 - } \hfill{Freiwillige Feuerwehr Gelting} \\
  \sep
  \textbf{2017 - } \hfill{Tischtennis BCF Wolfratshausen} \\
  \sep
  \textbf{2017 - } \hfill{Tischtennis SV Gelting}
\end{document}
