\documentclass[12pt]{article}

\newdimen\mymargin
\setlength{\mymargin}{1.0in}

\usepackage[margin=\mymargin]{geometry}
\usepackage{xcolor}
\usepackage{ulem}
\usepackage{microtype}
\usepackage{babel}
\usepackage{tabularx}

% title
\title{\vspace{-96pt}\Huge Lebenslauf}
\author{\Huge Nicholas Schweitzer}
\date{}

% link and section macros
\newcommand{\link}[1]{{\color{blue}\underline{#1}}}
\newcommand{\sect}[1]{
  {
    \vspace{6pt}
    \section*{
      \fontsize{18}{0}\selectfont
      \hspace{-12pt}
      \vspace{-12pt}
      #1
    }
  }
}

% page formatting options
\setlength\parindent{0pt}
\setlength\parskip{6pt}
\pagenumbering{gobble}

% other commands
\newcommand{\sep}{{\color{gray}\vspace{-12pt}\hrule}}
\def\labelitemi{--}
\newcommand{\ask}[1]{{\color{red}#1}}

% main document
\begin{document}
\maketitle
\vspace{-32pt}

% \section*{Motivation}
% Ich bin Sch{\"u}ler in die 10. Klasse und bin in ein Praktikum bei DLR entweder in Oberpfaffenhofen oder Weilheim im Bereich von Aerodynamik und Raumfahrt interessiert. Technologie, Physik und Mathematik fand ich immer interessant und ich hoffe durch ein Praktikum bei DLR mein Wissen zu erweitern und praktische Erfahrung zu gewinnen sowie mit anderen {\"u}ber meine Interessen reden.

\sect{Pers{\"o}nliche Daten}

\textbf{Geburt} \hfill{27.09.2003} \, \\
\sep
\textbf{Email} \hfill{nicholas\_schweitzer@mis-munich.de} \, \\
\sep
\textbf{Handy / Festnetz} \hfill{0176 8053 1248 / 08171 238 1130} \, \\
\sep
\textbf{Adresse} \hfill{Am Haken 2, 82538 Geretsried \, }\\
\sep

\sect{Profil}

\textbf{Sch{\"u}ler der 10. Klasse an der Munich International School; angestrebter
  abschluss: International Baccalaureate im Juni 2022}

\begin{itemize}
  \itemsep0pt

\item Schulischer Schwerpunkt in Mathematik und Physik mit Kursen in \textit{\glqq
  Advanced Mathematics\grqq} und \textit{\glqq Extended Science\grqq}

\item Programmieren seit 7 Jahren als Hobby mit fortgeschrittenen Kentnissen in
  Python und Rust

\item Musik und Sport: Gitarre seit 10 Jahren als intensives Hobby,
  Tischtennis Vereinsspieler

\item Sonstige Interessen: Freiwillige Feuerwehr und Sportklettern

\end{itemize}
\vspace{-24pt}

\sect{Schullaufbahn}
\vspace{6pt}

\hspace{-\mymargin}\begin{tabularx}{\paperwidth}{p{\dimexpr0.12\linewidth}|p{\linewidth}}
  \textbf{Seit 2014} & \textbf{Middle und Senior School der Munich International School} \\

  & \parbox{\linewidth}{
    \vspace{6pt}      % I don't know why \parskip doesn't work here

    Fachliche Schwerpunkte und zus{\"a}tzliche Kurse: \glqq Advanced Math\grqq\,
    (Fortgeschrittene Mathematik), \glqq Science Extension\grqq\, (erweiterte
    Physik, Chemie und Biologie, 3 Stunden pro Woche) und Musik, 3 Stunden pro
    Woche
    \vspace{6pt}      % I don't know why \parskip doesn't work here
    
    \glqq Head of School's List\grqq \,in die 5., 6., 7., 8. und 9. Klasse
    (Auszeichnung f{\"u}r die Jahrgangsbesten mit einer Durchscnittsnote von
    mindestens 6,5 auf einer Skala von 0 bis 7)
    \vspace{6pt}      % I don't know why \parskip doesn't work here
    
    Schuljahrbegleitendes \textit{\glqq Personal project\grqq}:
    Programmieren einen 3D-Renderer in Rust mit Fokus auf effiziente Nutzung des
    Grafikprozessors und Memory-Safety
    \vspace{6pt}      % I don't know why \parskip doesn't work here
    
    Programmierung einer Live-Demonstration f{\"u}r eine Veranstaltung mit
    Eltern und Sponsoren der Schule: virtuelles bewegungsbesteuertes Instrument
    (Teilnehmer k{\"o}nnten durch Bewegung vor einem Green-Screen Ton und
    Lautst{\"a}rke steuern)
  }
  \\
  & \\[-6pt]
  \hline
  & \\[-6pt]
  
  \textbf{\hbox{2011-2014}} & \textbf{Dreij{\"a}hrige Segelreise von der US Ostk{\"u}ste nach Australien} \\

  & \parbox{\linewidth}{
    \vspace{6pt}      % I don't know why \parskip doesn't work here

    Homeschooling durch die Eltern an Bord eines Segelbootes (2. - 4. Klasse,
    ausgew{\"a}hlte Schulprojekte unter \link{namaniatsea.org/nicky-page})
    \vspace{6pt}      % I don't know why \parskip doesn't work here

    Unterst{\"u}tzung meiner Eltern bei Navigation und Position-plotting
  } \\
\end{tabularx}

\pagebreak

\sect{Kenntnisse}
\textbf{Sprachen} \hfill{Englisch, Muttersprache, und Deutsch, flie{\ss}end} \\
\hspace*{\fill}Spanisch set 4 Jahren als Fremdsprache \\
\sep

\textbf{Programmiersprachen} \\
\textit{Sehr gut} \hfill{Python, Rust} \\
\sep
\textit{Gut} \hfill{C/C++, Shell (bash/zsh), Lua, \LaTeX} \\
\sep
\textit{Grundkentnisse} \hfill{Lisp, Perl, Intel-Assembly (NASM), HTML} \\
\sep

{
  \small
  \textbf{Github-Repos} \hfill{3D-Renderer, Rust: \link{github.com/cynic64/render-engine}} \\
  \hspace*{\fill} Bewegungsbesteuerter Theremin, Python: \link{github.com/cynic64/theremin} \\
  \hspace*{\fill} Window-Manager Konfiguration, Lua:
  \link{github.com/cynic64/awesome-configs} \\
}

\sep

\textbf{Software Anwendungen} \\
\textit{Sehr gut} \hfill{MS Word, Emacs und Vi Editoren, Blender 3D Modelling Software} \\
\sep
\textit{Gut} \hfill{MS Excel, MS PowerPoint, Git, Linux-Administration} \\
\sep

\sect{Mitgliedschaften}

\textbf{Seit 2018} \hfill{Freiwillige Feuerwehr Gelting} \\
\sep
\textbf{Seit 2017} \hfill{Tischtennis BCF Wolfratshausen} \\
\sep
\textbf{Seit 2017} \hfill{Tischtennis SV Gelting}

\end{document}

